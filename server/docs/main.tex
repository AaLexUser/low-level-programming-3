\documentclass[12pt,onecolumn]{article}
\usepackage[utf8]{inputenc} % UTF8 input encoding
\usepackage[T2A]{fontenc}   % T2A font encoding for Cyrillic script
\usepackage[russian]{babel} % Russian language support
\usepackage{listings}
\usepackage{float}
\usepackage{mathtools}
\everymath{\displaystyle}
\usepackage{listings} 
\usepackage[usenames]{color}
\usepackage{geometry}
\usepackage{verbatim}
\newcommand{\nparagraph}[1]{\paragraph{#1}\mbox{}\\}
\geometry{
  a4paper,
  top=20mm, 
  right=25mm, 
  bottom=20mm, 
  left=20mm
}

\lstdefinestyle{C}{language=C, 
  basicstyle=\small\ttfamily,
  commentstyle=\color{cyan},
  stringstyle=\color{magenta}\ttfamily,
  keywordstyle=\color{blue},
  numbers=left,
  numberstyle=\scriptsize,
  numbersep=5pt,
  frame=single,
  breaklines=true,
  breakatwhitespace=true,
  showstringspaces=false,
  tabsize=4,
  inputencoding=utf8,
  extendedchars=true,
  literate={а}{{\selectfont\char224}}1
          {б}{{\selectfont\char225}}1
          {в}{{\selectfont\char226}}1
          {г}{{\selectfont\char227}}1
          {д}{{\selectfont\char228}}1
          {е}{{\selectfont\char229}}1
          {ё}{{\"e}}1
          {ж}{{\selectfont\char230}}1
          {з}{{\selectfont\char231}}1
          {и}{{\selectfont\char232}}1
          {й}{{\selectfont\char233}}1
          {к}{{\selectfont\char234}}1
          {л}{{\selectfont\char235}}1
          {м}{{\selectfont\char236}}1
          {н}{{\selectfont\char237}}1
          {о}{{\selectfont\char238}}1
          {п}{{\selectfont\char239}}1
          {р}{{\selectfont\char240}}1
          {с}{{\selectfont\char241}}1
          {т}{{\selectfont\char242}}1
          {у}{{\selectfont\char243}}1
          {ф}{{\selectfont\char244}}1
          {х}{{\selectfont\char245}}1
          {ц}{{\selectfont\char246}}1
          {ч}{{\selectfont\char247}}1
          {ш}{{\selectfont\char248}}1
          {щ}{{\selectfont\char249}}1
          {ъ}{{\selectfont\char250}}1
          {ы}{{\selectfont\char251}}1
          {ь}{{\selectfont\char252}}1
          {э}{{\selectfont\char253}}1
          {ю}{{\selectfont\char254}}1
          {я}{{\selectfont\char255}}1
          {А}{{\selectfont\char192}}1
          {Б}{{\selectfont\char193}}1
          {В}{{\selectfont\char194}}1
          {Г}{{\selectfont\char195}}1
          {Д}{{\selectfont\char196}}1
          {Е}{{\selectfont\char197}}1
          {Ё}{{\"E}}1
          {Ж}{{\selectfont\char198}}1
          {З}{{\selectfont\char199}}1
          {И}{{\selectfont\char200}}1
          {Й}{{\selectfont\char201}}1
          {К}{{\selectfont\char202}}1
          {Л}{{\selectfont\char203}}1
          {М}{{\selectfont\char204}}1
          {Н}{{\selectfont\char205}}1
          {О}{{\selectfont\char206}}1
          {П}{{\selectfont\char207}}1
          {Р}{{\selectfont\char208}}1
          {С}{{\selectfont\char209}}1
          {Т}{{\selectfont\char210}}1
          {У}{{\selectfont\char211}}1
          {Ф}{{\selectfont\char212}}1
          {Х}{{\selectfont\char213}}1
          {Ц}{{\selectfont\char214}}1
          {Ч}{{\selectfont\char215}}1
          {Ш}{{\selectfont\char216}}1
          {Щ}{{\selectfont\char217}}1
          {Ъ}{{\selectfont\char218}}1
          {Ы}{{\selectfont\char219}}1
          {Ь}{{\selectfont\char220}}1
          {Э}{{\selectfont\char221}}1
          {Ю}{{\selectfont\char222}}1
          {Я}{{\selectfont\char223}}1
}

\begin{document}
\setcounter{tocdepth}{4}
\begin{center}
    Федеральное государственное автономное образовательное учреждение высшего образования "Национальный Исследовательский Университет ИТМО"\\ 
    Мегафакультет Компьютерных Технологий и Управления\\
    Факультет Программной Инженерии и Компьютерной Техники \\
    \includegraphics[scale=0.3]{image/itmo.jpg} % нужно закинуть картинку логтипа в папку с отчетом
\end{center}
\vspace{1cm}


\begin{center}
    \textbf{Лабораторная №1}\\
    по дисциплине\\
    \textbf{'Низкоуровневое программирование'}
\end{center}

\vspace{2cm}

\begin{flushright}
  Выполнил Студент  группы P33102\\
  \textbf{Лапин Алексей Александрович}\\
  Преподаватель: \\
  \textbf{Кореньков Юрий Дмитриевич}\\
\end{flushright}

\vspace{6cm}
\begin{center}
    г. Санкт-Петербург\\
    2023г.
\end{center}

\newpage
\tableofcontents
\newpage

\section{Цель:}
Выданный вариант - 5 (реляционные таблицы, отображение файла)
Целью является создать модуль, реализующий хранение в одном файле данных информации общим объёмом от 10GB в виде реляционных таблиц.
\section{Задачи:}
\begin{enumerate}
  \item Модуль работающий с файлом на низком уровне(маппинг, синхронизация, чтение, запись и п.т) с одной страницей(file);
  \item Модуль, который абстрагируется от низкоуровневых операций и предоставляет более удобный интерфейс для записи и чтения, а также осуществление кеширования страниц(caching);
  \item Модуль, который еще больше абстрагируется от низкоуровневых операций и добавляет возможность переиспользовать страницы, после их удаления(pager);
  \item Модуль, который соединяет страницы в список, для работы не с физическими страницами по 4кб, а с виртуальными условно бесконечными страницами. (linked-pages)
  \item Модуль, который строит поверх linked-pages pool страниц, блоки которого можно выдавать за O(1). (page-pool)
  \item Модуль, который расширяет блоки в page-pool, связывая их, чтобы можно было работать не с физическими блоками фиксированного размера, а с виртуальными условно бесконечными блоками. (linked-blocks)
  \item Модули, создания схемы таблицы и самих таблиц
  \item Модули, хранения метаинформации о существующих таблицах, строках переменной длинны и т.п.
  \item Операции над таблицами, вставка, удаление, обновление, выборка, соединение, сравнение rows и т.п.
\end{enumerate}
\subsection{Описание структур данных для представления данных в памяти:}


\begin{lstlisting}[style=C]
  /* Файл */
typedef struct file {
  char *filename;
  int fd;
  void *cur_mmaped_data;
  off_t cur_page_offset;
  off_t file_size;
  int64_t max_page_index;
} file_t;

/* Кеширование */
typedef struct caching{
    file_t file;
    size_t size, used, max_used, capacity;
    uint32_t* usage_count;
    time_t* last_used;
    void** cached_page_ptr;
    char* flags;
} caching_t;

/* Переиспользование страниц */
typedef struct pager{
    caching_t ch;
    int64_t deleted_pages; // index of parray page with deleted pages
} pager_t;

/* Связанные страницы */
typedef struct linked_page{
    int64_t next_page;
    int64_t page_index;
    int64_t mem_start;
} linked_page_t;

/* Пул страниц */

/* Чанк */
typedef struct chunk {
    linked_page_t lp_header;
    int64_t page_index;
    int64_t capacity;
    int64_t num_of_free_blocks;
    int64_t num_of_used_blocks;
    int64_t next;
    int64_t prev_page;
    int64_t next_page;
} chunk_t;

/* Менеджер пула */
typedef struct page_pool {
    linked_page_t lp_header;
    int64_t current_idx;
    int64_t head;
    int64_t tail;
    int64_t block_size;
    int64_t wait; // parray index
} page_pool_t;

/* Связанные блоки */
typedef struct linked_block{
    chblix_t next_block;
    chblix_t prev_block;
    chblix_t chblix;
    char flag;
    int64_t mem_start;
} linked_block_t;

  /* Поле схемы */
typedef struct field{
  linked_block_t lb_header;
  char name[MAX_NAME_LENGTH];
  datatype_t type;
  uint64_t size;
  uint64_t offset;
} field_t;

/* Схема таблицы */
typedef struct schema{
  page_pool_t ppl_header;
  int64_t slot_size;
} schema_t;

/* Таблица */ 
typedef struct table {
    page_pool_t ppl_header;
    int64_t schidx; //schema index
    char name[MAX_NAME_LENGTH];
} table_t;

/* Типы данных */
typedef enum datatype {DT_UNKNOWN = -1,
                       DT_INT = 0,
                       DT_FLOAT = 1,
                       DT_VARCHAR = 2,
                       DT_CHAR = 3,
                       DT_BOOL = 4} datatype_t;
\end{lstlisting}


\section{Описание работы:}
\subsection{Публичный интерфейс:}
\subsubsection{Таблица}
\begin{lstlisting}[style=C]

/**
* @brief       Initialize table and add it to the metatable
* @param[in]   db: pointer to db
* @param[in]   name: name of the table
* @param[in]   schema: pointer to schema
* @return      index of the table on success, TABLE_FAIL on failure
*/
table_t* tab_init(db_t* db, const char* name, schema_t* schema);
/**
 * @brief       Get row by value in column
 * @param[in]   db: pointer to db
 * @param[in]   table: pointer to the table
 * @param[in]   schema: pointer to the schema
 * @param[in]   field: pointer to the field
 * @param[in]   value: pointer to the value
 * @param[in]   type: type of the value
 * @return      chblix_t of row on success, CHBLIX_FAIL on failure
 */
chblix_t tab_get_row(db_t* db,
                    table_t* table,
                    schema_t* schema, 
                    field_t* field, 
                    void* value, 
                    datatype_t type);
/**
* @brief       Print table
* @param[in]   db: pointer to db
* @param[in]   table: index of the table
* @param[in]   schema: pointer to the schema
*/
void tab_print(db_t* db, 
            table_t* table,
            schema_t* schema);
/**
* @brief       Inner join two tables
* @param[in]   db: pointer to db
* @param[in]   left: pointer to the left table
* @param[in]   left_schema: pointer to the schema of the left table
* @param[in]   right: pointer to the right table
* @param[in]   right_schema: pointer to the schema of the right table
* @param[in]   join_field_left: join field of the left table
* @param[in]   join_field_right: join field of the right table
* @param[in]   name: name of the new table
* @return      pointer to the new table on success, NULL on failure
*/                                  
int64_t tab_join_on_field(
        db_t* db,
        int64_t leftidx,
        int64_t rightidx,
        const char* join_field_left,
        const char* join_field_right,
        const char* name);
/**
* @brief       Select row form table on condition
* @param[in]   db: pointer to db
* @param[in]   sel_table: pointer to table from which the selection is made
* @param[in]   sel_schema: pointer to schema of the table from which the selection is made
* @param[in]   select_field: the field by which the selection is performed
* @param[in]   name: name of new table that will be created
* @param[in]   condition: comparison condition
* @param[in]   value: value to compare with
* @param[in]   type: the type of value to compare with
* @return      pointer to new table on success, NULL on failure
*/
table_t* tab_select_op(db_t* db,
        table_t* sel_table,
        schema_t* sel_schema,
        field_t* select_field,
        const char* name,
        condition_t condition,
        void* value,
        datatype_t type);
/**
* @brief       Drop a table
* @param[in]   db: pointer to db
* @param[in]   table: pointer of the table
* @return      PPL_SUCCESS on success, PPL_FAIL on failure
*/
int tab_drop(db_t* db, table_t* table);

/**
 * @brief       Update row in table
 * @param[in]   db: pointer to db
 * @param[in]   table: pointer to table 
 * @param[in]   schema: pointer to schema 
 * @param[in]   field: pointer to field 
 * @param[in]   condition: comparison condition
 * @param[in]   value: value to compare with
 * @param[in]   type: the type of value to compare with 
 * @param[in]   row: pointer to new row which will replace the old one
 * @return 
 */
int tab_update_row_op(db_t* db,
                           table_t* table,
                           schema_t* schema,
                           field_t* field,
                           condition_t condition,
                           void* value,
                           datatype_t type,
                           void* row);

/**
 * @brief       Update element in table
 * @param[in]   db: pointer to db
 * @param[in]   tablix: index of the table
 * @param[in]   element: element to write
 * @param[in]   field_name: name of the element field
 * @param[in]   field_comp: name of the field compare with
 * @param[in]   condition: comparison condition
 * @param[in]   value: value to compare with
 * @param[in]   type: the type of value to compare with
 * @return      TABLE_SUCCESS on success, TABLE_FAIL on failure
 */

int tab_update_element_op(db_t* db,
                            int64_t tablix,
                            void* element,
                            const char* field_name,
                            const char* field_comp,
                            condition_t condition,
                            void* value,
                            datatype_t type);
/**
* @brief       Delete row from table
* @param[in]   db: pointer to db
* @param[in]   table: pointer to table
* @param[in]   schema: pointer to schema
* @param[in]   field_comp: pointer to field to compare
* @param[in]   condition: comparison condition
* @param[in]   value: value to compare with
* @return      TABLE_SUCCESS on success, TABLE_FAIL on failure
*/                          
int tab_delete_op(db_t* db,
                   table_t* table,
                   schema_t* schema,
                   field_t* comp,
                   condition_t condition,
                   void* value);
/**
* @brief       Insert a row
* @param[in]   table: pointer to table
* @param[in]   schema: pointer to schema
* @param[in]   src: source
* @return      chblix_t of row on success, CHBLIX_FAIL on failure
*/

chblix_t tab_insert(table_t* table, schema_t* schema, void* src)

/** @brief       Create a table on a subset of fields
*  @param[in]   db: pointer to db
*  @param[in]   table: pointer to table
*  @param[in]   schema: pointer to schema
*  @param[in]   fields: pointer to fields
*  @param[in]   num_of_fields: number of fields
*  @param[in]   name: name of the new table
*  @return      pointer to new table on success, NULL on failure
*/

table_t* tab_projection(db_t* db,
                        table_t* table,
                        schema_t* schema,
                        field_t* fields,
                        int64_t num_of_fields,
                        const char* name);

/**
* @brief       For each element specific column in a table
* @param[in]   table: pointer to the table
* @param[in]   chunk: chunk
* @param[in]   chblix: chblix of the row
* @param[in]   element: pointer to the element, must be allocated before calling this macro
* @param[in]   field: pointer to field of the element
*/

#define tab_for_each_element(table, chunk, chblix, element, field)
/**
 * @brief       For each element specific column in a table
 * @param[in]   table: pointer to the table
 * @param[in]   chunk: chunk
 * @param[in]   chblix: chblix of the row
 * @param[in]   row: pointer to the row, must be allocated before calling this macro
 * @param[in]   schema: pointer to the schema
 */

#define tab_for_each_row(table, chunk, chblix, row, schema)
\end{lstlisting}
\subsubsection{Схема}
\begin{lstlisting}[style=C]
  /**
  * @brief       Delete a schema
  * @param[in]   schidx: index of the schema
  * @return      SCHEMA_SUCCESS on success, SCHEMA_FAIL on failure
  */
 
 #define sch_delete(schidx)

 /**
 * @brief      Load field
 * @param[in]  schidx: index of the schema
 * @param[in]  fieldix: index of the field
 * @param[out] field: pointer to the field
 * @return     LB_SUCCESS on success, LB_FAIL on failure
 */

#define sch_field_load(schidx, fieldix, field)

/**
 * @brief      Update field
 * @param[in]  schidx: index of the schema
 * @param[in]  fieldix: index of the field
 * @param[out] field: pointer to the field
 * @return     LB_SUCCESS on success, LB_FAIL on failure
 */

#define sch_field_update(schidx, fieldix, field)

#define sch_add_int_field(schema, name) sch_add_field((schema), name, DT_INT, sizeof(int64_t))
#define sch_add_char_field(schema, name, size) sch_add_field((schema), name, DT_CHAR, size)
#define sch_add_varchar_field(schema, name) sch_add_field((schema), name, DT_VARCHAR, sizeof(vch_ticket_t))
#define sch_add_float_field(schema, name) sch_add_field((schema), name, DT_FLOAT, sizeof(float))
#define sch_add_bool_field(schema, name) sch_add_field((schema), name, DT_BOOL, sizeof(bool))

/**
 * @brief       For each field in a schema
 * @param[in]   sch: pointer to the schema
 * @param[in]   chunk: chunk
 * @param[in]   field: pointer to the field
 * @param[in]   chblix: chblix of the row
 * @param[in]   schidx: index of the schema
 */

#define sch_for_each(sch,chunk, field, chblix, schidx)
/**
 * @brief       Initialize a schema
 * @return      pointer to schema on success, NULL on failure
 */
void* sch_init(void);
/**
 * @brief       Add a field
 * @param[in]   schema: pointer to schema
 * @param[in]   name: name of the field
 * @param[in]   type: type of the field
 * @param[in]   size: size of the type
 * @return      SCHEMA_SUCCESS on success, SCHEMA_FAIL on failure
 */
int sch_add_field(schema_t* schema, const char* name, datatype_t type, int64_t size);
/**
 * @brief       Get a field
 * @param[in]   schema: pointer to the schema
 * @param[in]   name: name of the field
 * @param[out]  field: pointer to destination field
 * @return      SCHEMA_SUCCESS on success, SCHEMA_FAIL on failure
 */
int sch_get_field(schema_t* schema, const char* name, field_t* field);
/**
 * @brief       Delete a field
 * @warning     This function delete field, but dont touch offsets in other fields in schema.
 * @param[in]   schema: pointer to schema
 * @param[in]   name: name of the field
 * @return      SCHEMA_SUCCESS on success, SCHEMA_FAIL on failure
 */
int sch_delete_field(schema_t* schema, const char* name);
\end{lstlisting}
\subsubsection{База данных}
\begin{lstlisting}[style=C]
/**
 * @brief       Initialize database
 * @param[in]   filename: name of the file
 * @return      pointer to database on success, NULL on failure
 */
 void* db_init(const char* filename);
 /**
 * @brief       Close database
 * @return      DB_SUCCESS on success, DB_FAIL on failure
 */
int db_close(void);
/**
 * @brief       Drop database
 * @return      DB_SUCCESS on success, DB_FAIL on failure
 */
int db_drop(void);
\end{lstlisting}
\subsection{Модули:}
Директория core - содержит модули, строящие абстракции над работой с файлом, вводом-выводом и т.п. 
\begin{enumerate}
  \item file - осуществляет базовую работу с файлом, при использовании одной страницы.
  \item caching - кеширует страницы, освобождает их по мере заполнения оперативной памяти.
  \item pager - хранит очередь страниц, готовых для переиспользования.
  \item liked\_pages - реализация связанных страниц.
  \item page\_pool - реализация пула страниц.
  \item linked\_blocks - реализация связанных блоков.
\end{enumerate}
Директория backend - содержит модули, реализовывающие работу с реляционной базой данных.
\begin{enumerate}
  \item schema - реализация операций с схемой таблиц
  \item table\_base - базовые операции с таблицами (нужны для других модулей)
  \item matatab - таблица с метаданными (название и индекс страницы) других таблиц.
  \item varchar\_mgr - пул хранящий строки произвольной длинны и выдающий номерки для их поиска в нем.
  \item db - модуль выполняющий операции на старте (инициализация файла, метатаблицы, varchar\_mgr), и в завершении работы программы (закрытие файла, удаление файла). Также хранит индекс метатаблицы и тп.
  \item data-types - поддерживаемые типы данных.
  \item comparator - сравнение поддерживаемых типов.
\end{enumerate}
\section{Аспекты реализации:}
\begin{enumerate}
  \item Данные хранятся в страницах стандартного размера для операционной системы (обычно 4 кб). 
  \item Странницы кешируются и освобождаются по необходимости. Для этого создана map ключем в которой является номер страницы, а значением указатель на область памяти, статус-флаги, последние время использования, количество использования. Освобождение страницы из конца файла удаляет страницу. Освобождение страницы из середины файла ставит её в очередь не переиспользование при следующем запросе на аллокацию.
  \item Чтобы поддерживать возможность добавление единиц данных больше размера страницы, страницы могут расширяться путем связывания linkedlist с новыми страницами. Для обращения и удаления этих страниц используется номер первой входящей в список страницы.
  \item Чтобы выдавать элементы(строки) за O(1) на страницах построен pool. Pool состоит из chunks размером в одну связную страницу, chunks связаны между собой двусвязным linkedlist. При отдаче всех блоков из chunk мы переставляем указатель на следующий. Когда в chunk освобождается блок, то он ставится в очередь на использование, когда в текущем chunk закончится место. Если в chunk освобождены все строки, то страница полностью удаляется.
  \item Чтобы поддерживать возможность записывать в блоки данные произвольной длинны, блоки могут расшириться связываясь с помощью linkedlist.
  \item Схема таблицы построена поверх пула страниц, а поля являются блоками этого пула.
  \item Таблицы строятся поверх своих пулов страниц, строки являются блоками пула.
\end{enumerate}
\section{Результаты:}
\subsection{Операция вставки:}
\includegraphics[width=\textwidth]{image/insert.png}
\subsection{Операция выборки:}
\includegraphics[width=\textwidth]{image/select.png}
\subsection{Операция удаления:}
\includegraphics[width=\textwidth]{image/delete.png}
\subsection{Операция обновления:}
\includegraphics[width=\textwidth]{image/update.png}
\subsection{Размер файла:}
\includegraphics[width=\textwidth]{image/mem.png}
\subsection{Использование оперативной памяти:}
\includegraphics[width=\textwidth]{image/ram.png}
\section{Выводы:}
Как видно из графика вставка работает за O(1), скочки на графике связаны скорее всего с аллокацией дополнительных страниц, как для хранения строк, так и для хранения очередей. 

Выборка работает за O(n), так как мы проходим по всем строкам. Есть одна точка, которая сильно отличается от остальных, считаю её выбросом.

Удаление работает за O(n), так как мы проходим по всем строкам.

Обновление работает за O(n), так как мы проходим по всем строкам.

Размер файла растет линейно, так как мы добавляем новые страницы.

Использование оперативной памяти O(1), так как мы загружаем страницы, когда нам они нужны и освобождаем, когда уже нет.

\textbf{Что я узнал, чему научился:}
\begin{enumerate}
  \item Написал свою первую большую программу на C, больше прогрузился в язык, в работу с ним.
  \item Погрузился в то как работают базы данных изнури, написал свою.
  \item Узнал новые алгоритмы и структуры данных, научился сам их реализовывать.
  \item Научился работать с файлами, работать с памятью, работать с потоками ввода-вывода.
  \item Научился писать переносимые приложения под разные операционные системы: Linux, Windows, MacOS.
\end{enumerate}
\end{document}