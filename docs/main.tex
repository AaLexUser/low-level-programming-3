\documentclass[12pt,onecolumn]{article}
\usepackage[utf8]{inputenc} % UTF8 input encoding
\usepackage[T2A]{fontenc}   % T2A font encoding for Cyrillic script
\usepackage[russian]{babel} % Russian language support
\usepackage{listings}
\usepackage{float}
\usepackage{mathtools}
\everymath{\displaystyle}
\usepackage{listings} 
\usepackage[usenames]{color}
\usepackage{geometry}
\usepackage{verbatim}
\newcommand{\nparagraph}[1]{\paragraph{#1}\mbox{}\\}
\geometry{
  a4paper,
  top=20mm, 
  right=2mm, 
  bottom=20mm, 
  left=2mm
}

\lstdefinestyle{bash}{language=bash, 
  basicstyle=\small\ttfamily,
  commentstyle=\color{cyan},
  stringstyle=\color{magenta}\ttfamily,
  keywordstyle=\color{blue},
  numbers=left,
  numberstyle=\scriptsize,
  numbersep=5pt,
  frame=single,
  breaklines=true,
  breakatwhitespace=true,
  showstringspaces=false,
  tabsize=1,
  inputencoding=utf8,
  extendedchars=true,
  literate={а}{{\selectfont\char224}}1
          {б}{{\selectfont\char225}}1
          {в}{{\selectfont\char226}}1
          {г}{{\selectfont\char227}}1
          {д}{{\selectfont\char228}}1
          {е}{{\selectfont\char229}}1
          {ё}{{\"e}}1
          {ж}{{\selectfont\char230}}1
          {з}{{\selectfont\char231}}1
          {и}{{\selectfont\char232}}1
          {й}{{\selectfont\char233}}1
          {к}{{\selectfont\char234}}1
          {л}{{\selectfont\char235}}1
          {м}{{\selectfont\char236}}1
          {н}{{\selectfont\char237}}1
          {о}{{\selectfont\char238}}1
          {п}{{\selectfont\char239}}1
          {р}{{\selectfont\char240}}1
          {с}{{\selectfont\char241}}1
          {т}{{\selectfont\char242}}1
          {у}{{\selectfont\char243}}1
          {ф}{{\selectfont\char244}}1
          {х}{{\selectfont\char245}}1
          {ц}{{\selectfont\char246}}1
          {ч}{{\selectfont\char247}}1
          {ш}{{\selectfont\char248}}1
          {щ}{{\selectfont\char249}}1
          {ъ}{{\selectfont\char250}}1
          {ы}{{\selectfont\char251}}1
          {ь}{{\selectfont\char252}}1
          {э}{{\selectfont\char253}}1
          {ю}{{\selectfont\char254}}1
          {я}{{\selectfont\char255}}1
          {А}{{\selectfont\char192}}1
          {Б}{{\selectfont\char193}}1
          {В}{{\selectfont\char194}}1
          {Г}{{\selectfont\char195}}1
          {Д}{{\selectfont\char196}}1
          {Е}{{\selectfont\char197}}1
          {Ё}{{\"E}}1
          {Ж}{{\selectfont\char198}}1
          {З}{{\selectfont\char199}}1
          {И}{{\selectfont\char200}}1
          {Й}{{\selectfont\char201}}1
          {К}{{\selectfont\char202}}1
          {Л}{{\selectfont\char203}}1
          {М}{{\selectfont\char204}}1
          {Н}{{\selectfont\char205}}1
          {О}{{\selectfont\char206}}1
          {П}{{\selectfont\char207}}1
          {Р}{{\selectfont\char208}}1
          {С}{{\selectfont\char209}}1
          {Т}{{\selectfont\char210}}1
          {У}{{\selectfont\char211}}1
          {Ф}{{\selectfont\char212}}1
          {Х}{{\selectfont\char213}}1
          {Ц}{{\selectfont\char214}}1
          {Ч}{{\selectfont\char215}}1
          {Ш}{{\selectfont\char216}}1
          {Щ}{{\selectfont\char217}}1
          {Ъ}{{\selectfont\char218}}1
          {Ы}{{\selectfont\char219}}1
          {Ь}{{\selectfont\char220}}1
          {Э}{{\selectfont\char221}}1
          {Ю}{{\selectfont\char222}}1
          {Я}{{\selectfont\char223}}1
}

\begin{document}
\setcounter{tocdepth}{4}
\begin{center}
    Федеральное государственное автономное образовательное учреждение высшего образования "Национальный Исследовательский Университет ИТМО"\\ 
    Мегафакультет Компьютерных Технологий и Управления\\
    Факультет Программной Инженерии и Компьютерной Техники \\
    \includegraphics[scale=0.3]{image/itmo.jpg} % нужно закинуть картинку логтипа в папку с отчетом
\end{center}
\vspace{1cm}


\begin{center}
    \textbf{Лабораторная №3}\\
    по дисциплине\\
    \textbf{'Низкоуровневое программирование'}
\end{center}

\vspace{2cm}

\begin{flushright}
  Выполнил Студент  группы P33102\\
  \textbf{Лапин Алексей Александрович}\\
  Преподаватель: \\
  \textbf{Кореньков Юрий Дмитриевич}\\
\end{flushright}

\vspace{6cm}
\begin{center}
    г. Санкт-Петербург\\
    2023г.
\end{center}

\newpage
\tableofcontents
\newpage

\section{Цель:}
Выданный вариант задания: XML
На базе данного транспортного формата описать схему протокола обмена информацией и воспользоваться существующей библиотекой по выбору для реализации модуля, обеспечивающего его функционирование.
Протокол должен включать представление информации о командах создания, выборки, модификации и удаления данных в соответствии с данной формой, и результатах их выполнения.

Используя созданные в результате выполнения заданий модули, разработать в виде консольного приложения две программы: клиентскую и серверную части. Серверная часть – получающая по сети запросы и операции описанного формата и последовательно выполняющая их над файлом данных с помощью модуля из первого задания. Имя фала данных для работы получать с аргументами командной строки, создавать новый в случае
его отсутствия. Клиентская часть – в цикле получающая на стандартный ввод текст команд, извлекающая из него информацию о запрашиваемой операции с помощью модуля из второго задания и пересылающая её на сервер с помощью модуля для обмена информацией, получающая ответ и выводящая его в человеко-понятном виде в стандартный вывод.

\section{Порядок выполнения:}
\begin{enumerate}
  \item  {
    Изучить выбранную библиотеку
    \begin{enumerate}
      \item Библиотека должна обеспечивать сериализацию и десериализацию с валидацией в соответствии со схемой
      \item Предпочтителен выбор библиотек, поддерживающих кодогенерацию на основе схемы
      \item Библиотека может поддерживать передачу данных посредством TCP соединения
      \begin{itemize}
        \item Иначе, использовать сетевые сокеты посредством API ОС
      \end{itemize}
      \item Библиотека может обеспечивать диспетчеризацию удалённых вызовов
      \begin{itemize}
        \item Иначе, реализовать диспетчеризацию вызовов на основе информации о виде команды
      \end{itemize}
    \end{enumerate}
  }
  \item {
    На основе существующей библиотеки реализовать модуль, обеспечивающий взаимодействие
    \begin{enumerate}
      \item Описать схему протокола в поддерживаемом библиотекой формате
      \begin{itemize}
        \item Описание должно включать информацию о командах, их аргументах и результатах
        \item Схема может включать дополнительные сущности (например, для итератора)
      \end{itemize}
      \item Подключить библиотеку к проекту и сформировать публичный интерфейс модуля с использованием встроенных или сгенерированных структур данных используемой библиотеки
      \begin{itemize}
        \item Поддержать установление соединения, отправку команд и получение их результатов
        \item Поддержать приём входящих соединений, приём команд и отправку их результатов
      \end{itemize}
      \item Реализовать публичный интерфейс посредством библиотеки в соответствии с п1
    \end{enumerate}
  }

  \item {
    Реализовать серверную часть в виде консольного приложения
    \begin{enumerate}
      \item В качестве аргументов командной строки приложение принимает:
      \begin{itemize}
        \item Адрес локальной конечной точки для прослушивания входящих соединений
        \item Имя файла данных, который необходимо открыть, если он существует, иначе создать
      \end{itemize}
      \item Работает с файлом данных посредством модуля из задания 1
      \item Принимает входящие соединения и взаимодействует с клиентами посредством модуля из п2
      \item Поступающая информация о запрашиваемых операциях преобразуется из структур данных модуля взаимодействия к структурам данных модуля управления данными и наоборот
    \end{enumerate}
  }

  \item {
    Реализовать клиентскую часть в виде консольного приложения
    \begin{enumerate}
      \item В качестве аргументов командной строки приложение принимает адрес конечной точки для подключения
      \item Подключается к серверу и взаимодействует с ним посредством модуля из п2
      \item Читает со стандартного ввода текст команд и анализирует их посредством модуля из задания 2
      \item Преобразует результат разбора команды к структурам данных модуля из п2, передаёт их для обработки на сервер, возвращаемые результаты выводит в стандартный поток вывода
    \end{enumerate}
  }

  \item {
    Результаты тестирования представить в виде отчёта, в который включить:
    \begin{enumerate}
      \item В части 3 привести пример сеанса работы разработанных программ
      \item В части 4 описать решение, реализованное в соответствии с пп.2-4
      \item В часть 5 включить составленную схему п.2а
    \end{enumerate}
  }

\end{enumerate}
\section{Описание работы и реализации:}
\begin{itemize}
  \item Использовалась библиотека libxml2 для работы с xml.
  \item Сетевое взаимодействие реализовано посредством сокетов.
  \item Перед отправкой сообщения отправляется его длина, это справедливо и для requests, и для response.
  \item Ответ от сервера содержит в себе:
  \begin{itemize}
    \item Статус. ERROR – ошибка, OK – успешно
    \item Сообщение.
    \item Таблица(optional).
  \end{itemize}
\end{itemize}

\subsection{Серверная часть}
\begin{itemize}
  \item Сервер принимает на вход 2 аргумента: адрес и порт.
  \item После запуска сервер начинает слушать входящие соединения.
  \item При подключении клиента, сервер создает новый поток, в котором обрабатывает запросы клиента.
  \item При получении запроса, сервер десериализует его в ast, передает в модуль выполнения запроса.
  \item После получения ответа от модуля выполнения запроса, сервер сериализует ответ в xml и отправляет его клиенту.
\end{itemize}
\subsubsection{Прием клиентов и старт сервера}
\begin{lstlisting}[style=bash, language=C]
  int main(int argc, char **argv) {
    char *filename = DEFAULT_FILE;
    int port = DEFAULT_PORT;
    if (argc > 1) {
        port = atoi(argv[1]);
    }
    if (argc > 2) {
        filename = argv[2];
    }


    db_t *db = db_init(filename);

    int sock = init_socket(port);
    if (sock < 0) {
        return 1;
    }

    if (listen_socket(sock) < 0) {
        return 1;
    }

    logger(LL_INFO, __func__, "Listening on port %d", port);

    signal(SIGTERM, sigint_handler);

    FD_ZERO(&readfds); 
    FD_SET(sock, &readfds); 
    max_sd = sock; 


    struct timeval timeout;
    while (server_running) {
        fd_set tmpfds = readfds; 
        timeout.tv_sec = 1; // Set a timeout 1 second to allow periodic checks for server_running
        timeout.tv_usec = 0;

        int activity = select(max_sd + 1, &tmpfds, NULL, NULL, &timeout);

        if ((activity < 0) && (errno != EINTR)) {
            fprintf(stderr,"select error: %s", strerror(errno));
            break;
        }

        if (activity > 0) {
            // Check if the activity is on the server socket (new connection)
            if (FD_ISSET(sock, &tmpfds)) {
                int client = accept_socket(sock);
                if (client < 0) {
                    perror("accept failed");
                    continue;
                }

                logger(LL_INFO, __func__, "Client %d connected", client);

                // Add new socket to the array of sockets
                FD_SET(client, &readfds);
                if (client > max_sd) {
                    max_sd = client; 
                }

                struct handler_args *args = malloc(sizeof(struct handler_args));
                args->db = db;
                args->client = client;

                pthread_t client_thread;
                if (pthread_create(&client_thread, NULL, (void *) client_handler, args) != 0) {
                    perror("Failed to create thread");
                    server_running = false;
                    continue;
                }
                pthread_detach(client_thread);
            }
        }
    }

    for (int i = 0; i <= max_sd; i++) {
        if (FD_ISSET(i, &readfds)) {
            if(i != -1){
                close_socket(i);
            }
        }
    }
    db_close();
    return 0;
}
\end{lstlisting}

\section{Результаты работы программы}


\begin{lstlisting}[style=bash]
  ./lab3-client 127.0.0.1 8080
  > CREATE users WITH { name: string, lastname: string, student: bool, money: int, score: float}
  Message: Table users created successfully
  > CREATE group WITH { group_id: int, name: string }
  Message: Table group created successfully  
\end{lstlisting}

\begin{lstlisting}[style=bash]
  > INSERT { name: "Alex", lastname: "Lapin", student: true, money: 10000, score: 5.0 } INTO users
  Message: Row inserted successfully
  > INSERT { name: "Berman", lastname: "Clock", student: true, money: 232, score: 3.2 } INTO users
  Message: Row inserted successfully
  > INSERT { name: "Cristian", lastname: "Ronaldo", student: false, money: 100, score: 4.2 } INTO users
  Message: Row inserted successfully
  > INSERT { name: "Dima", lastname: "Koval", student: true, money: 2330, score: 2.1 } INTO users
  Message: Row inserted successfully
  > INSERT { name: "Egor", lastname: "Flagman", student: true, money: 100, score: 1.2 } INTO users
  Message: Row inserted successfully
  > INSERT { name: "Fedor", lastname: "Champion", student: false, money: 0, score: 5.0 } INTO users
  Message: Row inserted successfully
  > INSERT { group_id: 1, name: "P33102" } INTO group
  Message: Row inserted successfully
  > INSERT { group_id: 2, name: "M33103" } INTO group
  Message: Row inserted successfully
  > INSERT { group_id: 3, name: "G33104" } INTO group
  Message: Row inserted successfully
  > INSERT { group_id: 4, name: "Z33105" } INTO group
  Message: Row inserted successfully
  > INSERT { group_id: 5, name: "K33106" } INTO group
  Message: Row inserted successfully
\end{lstlisting}

\begin{lstlisting}[style=bash]
  > FOR u IN users FILTER u.money > 100 return u
  Message: Selected successfully
  Table:
  name    lastname  student  money   score          
  Alex    Lapin     true     10000   5.000000       
  Berman  Clock     true     232     3.200000       
  Dima    Koval     true     2330    2.100000


  > FOR u IN users FILTER u.money == 100 FOR g IN group FILTER g.group_id == 3 RETURN MERGE(u,g)
  Message: Selected successfully
  Table:
  name     lastname student money score     group_id  name           
  Cristian Ronaldo  false   100   4.200000  3         G33104         
  Egor     Flagman  true    100   1.200000  3         G33104

  
\end{lstlisting}

\begin{lstlisting}[style=bash]
  > CREATE players WITH { username: string, player: bool, cash: int, score: float}                                                             
  Message: Table players created successfully                                                                                                  
  > DROP players                                                                                                                               
  Message: Table players dropped successfully                                                                                                  
  > FOR p in players RETURN p                                                                                                                  
  Message: Failed to find table players  
\end{lstlisting}

\begin{lstlisting}[style=bash]
  > FOR u IN users RETURN u
  Message: Selected successfully
  Table:
  name            lastname        student         money           score          
  Alex            Lapin           true            10000           5.000000       
  Berman          Clock           true            232             3.200000       
  Cristian        Ronaldo         false           100             4.200000       
  Dima            Koval           true            2330            2.100000       
  Egor            Flagman         true            100             1.200000       
  Fedor           Champion        false           0               5.000000       
  > FOR u IN users FILTER u.name == "Berman" UPDATE u WITH { score: 5.0 } IN users
  Message: Updated successfully
  > FOR u IN users RETURN u
  Message: Selected successfully
  Table:
  name            lastname        student         money           score          
  Alex            Lapin           true            10000           5.000000       
  Berman          Clock           true            232             5.000000       
  Cristian        Ronaldo         false           100             4.200000       
  Dima            Koval           true            2330            2.100000       
  Egor            Flagman         true            100             1.200000       
  Fedor           Champion        false           0               5.000000

  > FOR u IN users RETURN u
  Message: Selected successfully
  Table:
  name            lastname        student         money           score          
  Alex            Lapin           true            10000           5.000000       
  Berman          Clock           true            232             5.000000       
  Cristian        Ronaldo         false           100             4.200000       
  Dima            Koval           true            2330            2.100000       
  Egor            Flagman         true            100             1.200000       
  Fedor           Champion        false           0               5.000000       
  > FOR u IN users FILTER u.score < 5.0 REMOVE u IN users
  Message: Removed successfully
  > FOR u IN users RETURN u
  Message: Selected successfully
  Table:
  name            lastname        student         money           score          
  Alex            Lapin           true            10000           5.000000       
  Berman          Clock           true            232             5.000000       
  Fedor           Champion        false           0               5.000000
  > FOR u IN users FILTER u.money > 1000 && u.student == true RETURN u
  Message: Selected successfully
  Table:
  name            lastname        student         money           score          
  Alex            Lapin           true            10000           5.000000
  > FOR u IN users FILTER u.money > 1000 || u.student == false RETURN u
  Message: Selected successfully
  Table:
  name            lastname        student         money           score          
  Alex            Lapin           true            10000           5.000000       
  Fedor           Champion        false           0               5.000000 
\end{lstlisting}


\section{Валидация xml схемы:}
\subsection{response.xsd}
\lstinputlisting[style=bash, language=XML]{../client/response.xsd}

\subsection{request.xsd}
\lstinputlisting[style=bash, language=XML]{../server/src/request.xsd}

\section{Сравнение с postgresql используя базу данных Northwind}
\subsection{Запрос к моей бд:}
\begin{lstlisting}[style=bash]
  > FOR o IN orders FILTER (o.customer_id == "HUNGO" OR o.customer_id == "SAVEA") AND o.freight > 100.0 FOR e IN employees FILTER e.employee_id == o.employee_id RETURN MERGE{e.first_name, e.last_name, o.customer_id, o.freight}
  Message: Selected successfully
  Table:
  first_name      last_name       customer_id     freight        
  Michael         Suyama          HUNGO           168.22         
  Anne            Dodsworth       SAVEA           214.27         
  Margaret        Peacock         HUNGO           124.12         
  Nancy           Davolio         SAVEA           126.56         
  Laura           Callahan        SAVEA           140.26         
  Michael         Suyama          SAVEA           367.63         
  Steven          Buchanan        SAVEA           200.24         
  Nancy           Davolio         SAVEA           544.08         
  Laura           Callahan        SAVEA           107.46         
  Anne            Dodsworth       HUNGO           142.33         
  Andrew          Fuller          SAVEA           352.69         
  Robert          King            SAVEA           388.98         
  Anne            Dodsworth       HUNGO           296.43         
  Michael         Suyama          HUNGO           220.31         
  Nancy           Davolio         SAVEA           167.05         
  Janet           Leverling       SAVEA           232.55         
  Nancy           Davolio         SAVEA           116.13         
  Andrew          Fuller          HUNGO           580.91         
  Andrew          Fuller          SAVEA           657.54         
  Nancy           Davolio         SAVEA           211.22         
  Margaret        Peacock         SAVEA           141.16         
  Robert          King            SAVEA           830.75         
  Michael         Suyama          SAVEA           227.22         
  Robert          King            SAVEA           400.81         
  Janet           Leverling       HUNGO           603.54         
  Margaret        Peacock         SAVEA           487.57         
  Michael         Suyama          SAVEA           252.49        
\end{lstlisting}
\subsection{Запрос к postgresql:}
\begin{lstlisting}[style=bash]
  SELECT e.first_name, e.last_name, o.customer_id, o.freight FROM orders AS o
  JOIN employees AS e ON o.employee_id = e.employee_id
  WHERE (o.customer_id = 'HUNGO' OR o.customer_id = 'SAVEA') AND o.freight > 100.0;
  "first_name"	"last_name"	"customer_id"	"freight"
  "Michael"	"Suyama"	"HUNGO"	168.22
  "Anne"	"Dodsworth"	"SAVEA"	214.27
  "Margaret"	"Peacock"	"HUNGO"	124.12
  "Nancy"	"Davolio"	"SAVEA"	126.56
  "Laura"	"Callahan"	"SAVEA"	140.26
  "Michael"	"Suyama"	"SAVEA"	367.63
  "Michael"	"Suyama"	"SAVEA"	252.49
  "Steven"	"Buchanan"	"SAVEA"	200.24
  "Nancy"	"Davolio"	"SAVEA"	544.08
  "Laura"	"Callahan"	"SAVEA"	107.46
  "Anne"	"Dodsworth"	"HUNGO"	142.33
  "Andrew"	"Fuller"	"SAVEA"	352.69
  "Robert"	"King"	"SAVEA"	388.98
  "Anne"	"Dodsworth"	"HUNGO"	296.43
  "Michael"	"Suyama"	"HUNGO"	220.31
  "Nancy"	"Davolio"	"SAVEA"	167.05
  "Janet"	"Leverling"	"SAVEA"	232.55
  "Margaret"	"Peacock"	"SAVEA"	487.57
  "Nancy"	"Davolio"	"SAVEA"	116.13
  "Janet"	"Leverling"	"HUNGO"	603.54
  "Andrew"	"Fuller"	"HUNGO"	580.91
  "Robert"	"King"	"SAVEA"	400.81
  "Andrew"	"Fuller"	"SAVEA"	657.54
  "Nancy"	"Davolio"	"SAVEA"	211.22
  "Margaret"	"Peacock"	"SAVEA"	141.16
  "Robert"	"King"	"SAVEA"	830.75
  "Michael"	"Suyama"	"SAVEA"	227.22
\end{lstlisting}

Видно, что результаты запросов совпадают.

\subsection{Запрос к моей бд:}
\begin{lstlisting}[style=bash]
  FOR o IN orders FILTER o.customer_id == "RICSU" AND o.freight > 100.0 FOR e IN employees FILTER e.employee_id == o.employee_id RETURN MERGE{e.first_name, e.last_name, o.freight}
\end{lstlisting}
\includegraphics{image/mydb1.png}
\subsection{Запрос к postgresql:}
\begin{lstlisting}[style=bash]
  SELECT e.first_name, e.last_name, o.freight FROM orders AS o
  JOIN employees AS e ON o.employee_id = e.employee_id
  WHERE o.customer_id = 'RICSU' AND o.freight > 100.0;
\end{lstlisting}
\includegraphics{image/pg1.png}\\
Видно, что результаты запросов совпадают.

\section{Выводы:}

\textbf{Что я узнал, чему научился:}
\begin{enumerate}
    \item Реализовать поддержку xml схемы в libxml2
    \item Реализовать сетевое взаимодействие с помощью сокетов
    \item Работать с xml сообщениями согласно схемам
    \item Сериализовывать и десериализовывать xml сообщения в структуры данных и наоборот
\end{enumerate}
\end{document}